%%%%%%%%%%%%%%%%%%%%%%%%%%%%%%%%%%%%%%%%%
% Beamer Presentation
% LaTeX Template
% Version 1.0 (10/11/12)
%
% This template has been downloaded from:
% http://www.LaTeXTemplates.com
%
% License:
% CC BY-NC-SA 3.0 (http://creativecommons.org/licenses/by-nc-sa/3.0/)
%
%%%%%%%%%%%%%%%%%%%%%%%%%%%%%%%%%%%%%%%%%

%----------------------------------------------------------------------------------------
%	PACKAGES AND THEMES
%----------------------------------------------------------------------------------------

\documentclass{beamer}

\usepackage[utf8]{inputenc}
\usepackage[T1]{fontenc}

\mode<presentation> {

% The Beamer class comes with a number of default slide themes
% which change the colors and layouts of slides. Below this is a list
% of all the themes, uncomment each in turn to see what they look like.

%\usetheme{default}
%\usetheme{AnnArbor}
%\usetheme{Antibes}
%\usetheme{Bergen}
%\usetheme{Berkeley}
%\usetheme{Berlin}
%\usetheme{Boadilla}
%\usetheme{CambridgeUS}
%\usetheme{Copenhagen}
%\usetheme{Darmstadt}
%\usetheme{Dresden}
%\usetheme{Frankfurt}
%\usetheme{Goettingen}
%\usetheme{Hannover}
%\usetheme{Ilmenau}
%\usetheme{JuanLesPins}
%\usetheme{Luebeck}
\usetheme{Madrid}
%\usetheme{Malmoe}
%\usetheme{Marburg}
%\usetheme{Montpellier}
%\usetheme{PaloAlto}
%\usetheme{Pittsburgh}
%\usetheme{Rochester}
%\usetheme{Singapore}
%\usetheme{Szeged}
%\usetheme{Warsaw}

% As well as themes, the Beamer class has a number of color themes
% for any slide theme. Uncomment each of these in turn to see how it
% changes the colors of your current slide theme.

%\usecolortheme{albatross}
%\usecolortheme{beaver}
%\usecolortheme{beetle}
%\usecolortheme{crane}
%\usecolortheme{dolphin}
%\usecolortheme{dove}
%\usecolortheme{fly}
%\usecolortheme{lily}
%\usecolortheme{orchid}
%\usecolortheme{rose}
%\usecolortheme{seagull}
\usecolortheme{seahorse}
%\usecolortheme{whale}
%\usecolortheme{wolverine}

%\setbeamertemplate{footline} % To remove the footer line in all slides uncomment this line
%\setbeamertemplate{footline}[page number] % To replace the footer line in all slides with a simple slide count uncomment this line

%\setbeamertemplate{navigation symbols}{} % To remove the navigation symbols from the bottom of all slides uncomment this line
}

\usepackage{graphicx} % Allows including images
\usepackage{booktabs} % Allows the use of \toprule, \midrule and \bottomrule in tables

%----------------------------------------------------------------------------------------
%	TITLE PAGE
%----------------------------------------------------------------------------------------

\title[Ansible]{Ansible ile Django Deployment} % The short title appears at the bottom of every slide, the full title is only on the title page

\author{Cihan Okyay} % Your name
\institute[pyistanbul] % Your institution as it will appear on the bottom of every slide, may be shorthand to save space
{
Python Istanbul \\ % Your institution for the title page
\medskip
\textit{@cihann, okyaycihan@gmail.com} % Your email address
}
\date{\today} % Date, can be changed to a custom date

\begin{document}

\begin{frame}
\titlepage % Print the title page as the first slide
\end{frame}

\begin{frame}
\frametitle{Genel Bakış} % Table of contents slide, comment this block out to remove it
\tableofcontents % Throughout your presentation, if you choose to use \section{} and \subsection{} commands, these will automatically be printed on this slide as an overview of your presentation
\end{frame}

%----------------------------------------------------------------------------------------
%	PRESENTATION SLIDES
%----------------------------------------------------------------------------------------

%------------------------------------------------
\section{Ansible 101} % Sections can be created in order to organize your presentation into discrete blocks, all sections and subsections are automatically printed in the table of contents as an overview of the talk
\begin{frame}
{Nedir?}
\frametitle{Ansible 101}
\begin{itemize}
\item Konfigürasyon Yönetimi
\item SSH
\item Python
\item Açık Kaynak
\item YAML
\end{itemize}

\end{frame}

\begin{frame}
\textbf{Kurulum}
\begin{itemize}
\item sudo apt-add-repository -y ppa:ansible/ansible
\item sudo apt-get update
\item sudo apt-get install -y ansible
\vspace{.25in}
\item pip install ansible
\end{itemize}

\vspace{0.25in}

\textbf{Komutlar}
\begin{itemize}
\item ansible all -m command -a whoami
\item ansible all -m ping
\end{itemize}
\end{frame}


\begin{frame}
\textbf{Modüller}
\begin{itemize}
\item apt/yum/pip
\item command/shell
\item file
\item copy
\item service
\item template
\end{itemize}

\vspace{0.25in}

\textbf{Cloud modülleri}
\begin{itemize}
\item EC2
\item Docker
\item Digital Ocean
\item ...
\end{itemize}
\end{frame}



\begin{frame}
\textbf{Playbooks}
\begin{itemize}
\item ansible-playbook playbook.yml
\end{itemize}

\vspace{0.25in}

\textbf{playbook.yml}

--\\
- hosts: all\\
    user: root\\
    sudo: yes\\
  
  tasks:\\
    - apt: name=nginx state=present

\end{frame}




\section{Django Deployment}

\begin{frame}[fragile]
\frametitle{Django Deployment}
\textbf{Neleri otomatize edebiliriz?}
\begin{itemize}
\item Sistem paket kurulumları
\item Python paketleri ve virtualenv
\item git clone
\item django modülü
\end{itemize}


\vspace{0.25in}


\textbf{Demo}

pyistanbul'un sitesini Ansible kullanarak Linux makine üzerinde koşturacağız. 

Kaynak kod için: https://github.com/cihann/pyistanbul-ansible


\end{frame}


%------------------------------------------------

\end{document} 
